% ----------------- Fuentes -----------------
\usepackage{fontspec}
\setmainfont{DejaVu Sans}
\setmonofont{DejaVu Sans Mono}
\usepackage{microtype}
\usepackage{setspace}
\setstretch{1.5}

% ----------------- Matemática -----------------
\usepackage{amsmath,amssymb}

% ----------------- Gráficos -----------------
\usepackage{graphicx}
\makeatletter
% Nunca exceder ancho de texto ni alto de página
\def\maxwidth{\ifdim\Gin@nat@width>\textwidth \textwidth \else \Gin@nat@width \fi}
\def\maxheight{\ifdim\Gin@nat@height>\textheight \textheight \else \Gin@nat@height \fi}
\makeatother
\setkeys{Gin}{width=\maxwidth,height=\maxheight,keepaspectratio}

% Centrar figuras creadas por pandoc
\usepackage{etoolbox}
\AtBeginEnvironment{figure}{\centering}

% Flotantes
\usepackage{float}
\renewcommand{\topfraction}{0.9}
\renewcommand{\textfraction}{0.05}
\renewcommand{\floatpagefraction}{0.8}

% Barreras por sección
\usepackage[section]{placeins}

% ----------------- Tablas compactas -----------------
\usepackage{array,booktabs,longtable}
% que ocupen todo el ancho disponible
\usepackage{calc}
\setlength{\LTleft}{0pt}
\setlength{\LTright}{0pt}
% reglas más pegadas (booktabs)
\setlength{\aboverulesep}{0pt}
\setlength{\belowrulesep}{0pt}
% tamaño y espaciados (más pequeño)
\newcommand{\TableTight}{%
  \scriptsize
  \setlength{\tabcolsep}{3pt}%
  \renewcommand{\arraystretch}{0.85}%
}
\AtBeginEnvironment{table}{\TableTight}
\AtBeginEnvironment{longtable}{\TableTight}
\AtBeginEnvironment{tabular}{\scriptsize}

% ----------------- Leyendas -----------------
\usepackage[font=footnotesize,labelfont=bf]{caption}
\captionsetup[table]{font=footnotesize,skip=4pt}
\captionsetup[figure]{font=footnotesize,skip=6pt}

% ----------------- Hipervínculos -----------------
\hypersetup{colorlinks=true, linkcolor=blue, urlcolor=blue}

% pandoc   --from=markdown+implicit_figures+link_attributes   00-reporte.md -o reporte.pdf   --pdf-engine=xelatex   -V geometry:top=0.67in -V geometry:bottom=0.67in -V geometry:left=0.85in -V geometry:right=0.85in   -H header.tex   --resource-path=.:images:../images